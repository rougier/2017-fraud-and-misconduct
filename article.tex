\documentclass[a4paper,10pt,onecolumn]{article}

% Wider margins (compared to default)
%\usepackage[margin=2.5cm]{geometry}
\usepackage[left=2.5cm,right=2.5cm,top=2.5cm,bottom=2.5cm]{geometry}


% English support (typography and hyphenation)
\usepackage[american]{babel}
\usepackage{csquotes}

% For math
\usepackage{amsmath}

% Unicode encoding
\usepackage[utf8]{inputenc}

% Color box
\usepackage{tikz}
\usepackage[most]{tcolorbox}


% Better default font (Libertine and Inconsolata)
\usepackage[ttscale=.875]{libertine}
\usepackage[scaled=0.96]{zi4}


% Biber
\usepackage[backend=biber,
            style=apa,
            % backref=true,
            % backrefstyle=three,
            maxcitenames=3,
            maxbibnames=99,
            apamaxprtauth=99,
            natbib=true]{biblatex}
\DeclareLanguageMapping{american}{american-apa}

% Figure & graphics
\usepackage{tikz}
\usepackage{wrapfig}
\usepackage{graphicx}

% Figure caption
\usepackage[labelsep=period]{caption}
\captionsetup[figure]{font=small,skip=0pt}
\renewcommand{\captionfont}{\small\sffamily}
\renewcommand{\captionlabelfont}{\small\sffamily\bfseries}

% Hyperref
\usepackage{xcolor}
\definecolor{blendedblue}{rgb}{0.2, 0.2, 0.6}
\definecolor{blendedred}{rgb}{0.8, 0.2, 0.2}
\usepackage[bookmarks=true,
            breaklinks=true,
            pdfborder={0 0 0},
            citecolor=blendedblue,
            colorlinks=true,
            linkcolor=blendedblue,
            urlcolor=blendedblue,
            citecolor=blendedblue,
            linktocpage=false,
            hyperindex=true,
            linkbordercolor=white]{hyperref}
\usepackage{hyperref}
\hypersetup{colorlinks=true}


% Bibliography file
\bibliography{article.bib}

% Headers
\usepackage{fancyhdr}
\pagestyle{fancy}
% \fancyhf{}
\rhead{\footnotesize \sf \today}
\lhead{\footnotesize \sf NP Rougier \textbullet~Ten Simple Rules for Fraud \& Misconduct}
\rfoot{\footnotesize \sf \thepage}
\cfoot{}
\lfoot{}

\makeatletter
\renewcommand{\maketitle}{\bgroup\setlength{\parindent}{0pt}
\begin{flushleft}
  \textbf{\huge\@title\\}
  \vspace{5mm}
  \@author
\end{flushleft}\egroup
}
\makeatother

\title{Ten Simple Rules for Fraud \& Misconduct}
\author{%
  \textbf{Nicolas P. Rougier}$^{1,2,3,*}$\\
  \begin{footnotesize}
    $^{1}$INRIA Bordeaux Sud-Ouest
          Talence, France
    $^{2}$Institut des Maladies Neurodégénératives,
          Université de Bordeaux, CNRS UMR 5293,
          Bordeaux, France
    $^{3}$LaBRI, Université de Bordeaux, Institut Polytechnique de Bordeaux,
          CNRS, UMR 5800, Talence, France\\
    $^{*}$Corresponding author:
          \href{mailto:Nicolas.Rougier@inria.fr}{Nicolas.Rougier@inria.fr}
  \end{footnotesize}
}

\date{\today}

\begin{document}
\maketitle

\begin{tikzpicture}[remember picture, overlay]
    \node[xshift=-2.85cm,yshift=-2.85cm] at (current page.north east) {
        \includegraphics[width=3cm]{stample}
    };
\end{tikzpicture}

%% \begin{figure}[h]
%%   \begin{center}
%%     \noindent \includegraphics[width=\textwidth]{fraud-kit}
%%   \end{center}
%%   \caption{My First Fraud Kit. Copyright (c) 2012 Aurich Lawson. Ars Technica
%%     \citep{timmer:2012}.}
%% \end{figure}


\begin{tcolorbox}[breakable, pad at break*=1mm,
    colback=yellow!5, arc=0pt, outer arc=0pt, boxrule=0pt, frame hidden]
\begin{small}
\noindent {\sf \textbf{Disclaimer.} We do not encourage scientific fraud nor
  misconduct. The goal of this provocative article is to alert the reader on a
  preoccupying situation in Science. Facing the Publish or Perish alternative,
  a number of researchers have decided to cross the Rubicon without the full
  knowledge of the dramatic consequences that may hurt Science and impact not
  only their scientific life but their actual life as well. If you're tempted
  to (even so slightly) beautify your results, keep in mind that the game is
  probably not worth the risks.}\par
\end{small}
\end{tcolorbox}

\section*{Introduction}

So, here we are! You've decided to join the dark side of Science. That's great!
You'll soon discover a brand new world of shiny and surprising results,
non-replicable experiments, fabricated data, funny statistics, fame and shame,
joy and sadness, and... jail. But you've made your choice (for whatever
reasons) and you better have to be well prepared because Science is strong on
the other side. Only a few years from now, fraud and misconduct was a piece of
cake (\href{https://en.wikipedia.org/wiki/The_Turk}{Mechanical Turk},
\href{https://en.wikipedia.org/wiki/Charles_Redheffer}{Perpetual motion
  machine}, \href{https://en.wikipedia.org/wiki/Great_Moon_Hoax}{Life on Moon},
\href{https://en.wikipedia.org/wiki/Piltdown_Man}{Piltdown man},
\href{https://en.wikipedia.org/wiki/Water_memory}{Water memory}). Problem is
that there are new players in town (\href{https://pubpeer.com}{PubPeer},
\href{http://retractionwatch.com}{RetractionWatch},
\href{https://forbetterscience.com}{For Better Science},
\href{http://blogs.discovermagazine.com/neuroskeptic/}{Neuroskeptic} to name
just a few) and they're pretty good at spotting and reporting fraudsters.
Furthermore, publishers have started to retaliate with high-tech tools. To
commit fraud or misconduct without getting caught in 2017 is kind of a
challenge and requires a real dedication to your task. But there will be always
people that think they're smarter than a whole community. And because we love
to have things done in the proper way, we're kind enough to give you some
simple rules to follow in your brand new career (see also \citep{timmer:2012}
for a set of complementary rules). However, results and consequences are not
guaranteed.

\section*{Rule 1: Misrepresent, falsify or fabricate your data}

In order to start a plain and deceitful scientific fraudster life, the very
first thing you need to gain expertise on is to learn how to convincingly
misrepresent, falsify or fabricate your data. If you're still hesitating at
embracing the dark side of Science, you may start with a slight
mis-representation to strengthen your hypothesis. Some classical techniques are
illustrated in the seventh rule of \citep{rougier:2014} but there are
actually \href{https://en.wikipedia.org/wiki/Misleading_graph}{many more
  techniques} to make a graph misleading \citep{wainer:1984,raschke:2008}. You
can also directly change your results in a post-production phase by using
software such as Photoshop or Gimp that makes it really easy to change pretty
much anything in a figure \citep{cromey:2012, hendricks:2011}. But be careful
of not making too obvious manipulations because an ever-growing number of
journals are now equipped with forensic software to spot image manipulation
\citep{white:2007, rossner:2004,hsu:2015} (see also
\href{http://journals.plos.org/ploscompbiol/s/figures#loc-image-manipulation}{PLoS
  image manipulation recommendation}). You can even test your image online
using \href{https://29a.ch/photo-forensics/#forensic-magnifier}{Forensically}
by Jonas Wagner. This should dissuade you from manipulating your image by
yourself.
%
%% \begin{wrapfigure}{r}{0.4\textwidth}
%% %  \vspace{-8mm}
%%   \begin{center}
%%     \includegraphics[width=0.4\textwidth]{datasaurus}
%%   \end{center}
%%   \caption{The Datasaurus set by \href{http://www.thefunctionalart.com/2016/08/download-datasaurus-never-trust-summary.html}{Alberto Cairo}. }
%%   \label{fig:datasaurus}
%% \end{wrapfigure}
%
Furthermore, none of the previous techniques will change the statistics of your
data and it might a better choice to falsify your data. Starting with real
data, you can change only a few points in order to change a non-significant
results into an astoundingly highly significant result as shown on the
\href{http://shinyapps.org/apps/p-hacker/}{p-hacker} application
\citep{schonbrodt:2015} or the interactive applet from
\citep{aschwanden:2015}. The advantage of falsifying real data is that changing
only a few points will still make your data looks good and not too much
suspicious. The disadvantage is that it requires some real data in the first
place and thus it requires to carry on some actual experiment (i.e. some
work). As a last resort, you can also fabricate the whole data set using a
specific program. For example, starting from a given X mean/SD, Y mean/SD, and
a correlation coefficient, you can generate a data set with these precise
statistics \citep{matejka:2017}. Do not pick the datasaurus set because if you
plot it, it might look suspicious for the alert reader. Finally, make sure to
have a backup story when people will start asking or investigating
how/when/where/how did you do perform the experiments to get these data. A
number of misconducts have been detected after just a few questions
\citep{vastag:2006}.


\section*{Rule 2: Hack your results}

%% \begin{wrapfigure}{l}{0.4\textwidth}
%%   \vspace{-5mm}
%%   \begin{center}
%%     \includegraphics[width=0.4\textwidth]{xkcd}
%%   \end{center}
%%   \caption{P-Values by XKCD (\url{https://xkcd.com/1478/}). This work is
%%     licensed under a CC-BY-NC 2.5 License.}
%%   \label{fig:xkcd}
%% \end{wrapfigure}
If you are reluctant at manipulating your data, you still have the option of
simply hacking your results in order to reach the significance level
(a.k.a. p-hacking). But before committing any scientific misconduct, are you
sure you need to do so? What is the p value of your NHST (Null Hypothesis
Significance Test)? Did you consider using one of the many qualification
proposed by
\href{https://mchankins.wordpress.com/2013/04/21/still-not-significant-2/}{Matthew
  Hankins} to report your results? Can't you use expression such as {\em nearly
  acceptable level of significance (p=0.06)}, {\em very closely brushed the
  limit of statistical significance (p=0.051)} or {\em weakly statistically
  significant (p=0.0557)}? It does not make sense, but in some cases, it might
be sufficient to convince a naive reviewer or reader. However, if you're
suffering from a bi-polar p-value disorder (term coined by
\href{http://daniellakens.blogspot.fr/2014/05/the-probability-of-p-values-as-function.html}{Daniel
  Lakens}), you'll want to reach the magical 0.05 threshold and it might
actually be much easier than you think because the correct usage of the
infamous p-value is hardly understood by a number of students, teachers or
researchers \citep{haller:2002,lecoutre:2003}. The current situation is
actually so bad that the American Statistical Association had to release a
statement to explain the correct usage \citep{wasserstein:2016,} last year and
Nature has issued as well some warning on the usage of the p-value
\citep{baker:2016}. Consequently, before hacking your results, it might be wise
to re-read your classics
\citep{simmons:2011,cumming:2012a,cumming:2012b,colquhoun:2014} if you want to
take full advantage of the misuse of p-value. Last, but not least, if you're in
psychology, you have to take extra-precaution because this domain has recently
raised concerns, especially when it comes to the unusual distribution of
p-values smaller than 0.05 \citep{hartgerink:2016,bakker:2012}. If you add the
statcheck variable into the equation \citep{nuijten:2015,epskamp:2016}, it
might be difficult to achieve a proper p-hacking of your results without being
noticed. Even though some people would call such results checking {\em
  methodological terrorism} \citep{finske:2016}, it might be time to reconsider
your commitment to the dark side of Science.


\section*{Rule 3: Copy/paste from others}

You have your (fake) results but you still need to publish your clickbait
article to gain fame (and tenure and professorship). Writing is a tedious task
and might represent a fair amount of work. Writing the state of the art in your
domain requires to actually read (what?) what your colleagues have been doing
over the past few years. It is a very time consuming task. But if one of these
colleagues wrote a nice introduction to the domain or a wonderful state of the
art, why bother writing a new one? Why not copy/paste what he/she has written
instead?  Plagiarism is the nuts and bolts of scientific misconduct
\citep{neuroskeptic:2017}, be it literal copying, substantial copying or
paraphrasing (see
\href{https://www.elsevier.com/editors/perk/plagiarism-complaints}{definitions
  from Elsevier} and the committee on publication ethics (COPE) procedure to
handle plagiarism in a
\href{https://www.elsevier.com/__data/assets/pdf_file/0005/72815/plagiarism-A.pdf}{submitted}
or
\href{https://www.elsevier.com/__data/assets/pdf_file/0020/72830/plagiarism-B_0.pdf}{published}
article). Of course, you cannot literally copy paste an excerpt of text from an
article to insert it in yours. Just imagine a situation where the reviewer
would also be the author of the text being reviewed. Consequently, if you
really intend to plagiarize, you'll need to transform the text drastically to
avoid detection by humans \citep{dorigo:2015,} and more importantly, to avoid
detection by dedicated software (e.g. iThenticate). These software are used by
universities and journals for tracking people just like you. And since their
database is really huge, plagiarism is harder and harder. Can you imagine than
self-plagiarism is not even allowed for a vast majority of journals? In the
end, you really have to improve or innovate in your techniques and to work very
hard \citep{long:2009}. But faced with such an adversarial nature of the
situation, to write your very own text might be an easier option.


\section*{Rule 4: Write your own peer-review}

%% \begin{wrapfigure}{r}{0.5\textwidth}
%%   \vspace{-5mm}
%%   \begin{center}
%%     \includegraphics[width=0.5\textwidth]{phd-comics}
%%   \end{center}
%%   \caption{Addressing reviewers comments. PhD Comics
%%     (\url{http://phdcomics.com}). Copyright (c) 2005 Jorge Cham.}
%% \end{wrapfigure}

Results and papers are ready and it is now time to submit. Depending on the
journal you target, you may still encounter some problem with the review
process. If reviewers are a bit too picky, they may ask you annoying questions,
be a bit suspicious, request more information, ask for a major revision and or
even reject your submission (can you imagine?). After all your hard work at
fabricating the data and plagiarized the article, that would be really
unfair. Fortunately, there is a simple solution. You can actually write your
own review! How is that possible you may ask? It is incredibly easy. At the
time of submission, you will be asked to give name of possible reviewers, just
give phony names with corresponding email addresses that will be redirected to
your mailbox. You will soon receive an invitation to review and you're then
free to write your own review and state how brilliant is your own work. Of
course, you'll have to write a review that looks like an actual review. If
you're Machiavellian, you will have introduced some factual errors in your
manuscript such that you can report them later in your review. Make sure also
to not send you review before the editor deadline because as
\href{https://arstechnica.com/science/2017/04/107-cancer-papers-retracted-due-to-peer-review-fraud/}{reported
  by Elizabeth Wager} \citep{stigbrand:2017}, one of the reasons the editors
spotted fake reviews was that reviewers responded just in time. However, editors and
publishers are now well aware of the scam \citep{ferguson:2014} and they have
taken counter-measures \citep{haug:2015}. For example, some of them do not offer
anymore the option to recommend unknown reviewer or if they do, the
recommendation is restricted to a list of certified reviewers. If you insist on
writing your own peer-review, you'll have to be creative and find new ways to
game the system.


\section*{Rule 5: Take advantage of predatory publishers}

If most journals will most likely reject your submission because you've applied
some of the rules above and get caught red-handed, you still have opportunities
to publish your results in one of the many predatory publishers available on
the internet and the internet only \citep{shen:2015}. Such predators will
publish just anything
(\href{https://en.wikipedia.org/wiki/International_Journal_of_Advanced_Computer_Technology#Publication_controversy}{``Get
  Me Off Your Fucking Mailing List''} in the International Journal of Advanced
Computer Technology (2005, 2014)) and you have a 100\% chance to have your
article published with a lighting fast review, less than 24h for some
journals. Finding a predatory publisher used to be easy thanks to the list
created and maintained by Jeffrey Beal. Even if some people (and publishers)
have judged this list to be controversial, it was incredibly useful for a lot
of people. Unfortunately, this list has closed just a few months ago
\citep{straumsheim:2017} and the
soon-to-be released replacement one will not be free \citep{silver:2017}. But
you can now take advantage of the
\href{http://thinkchecksubmit.org}{Think/Check/Submit} website that provides a
easy-to-use checklist that researchers can refer to when they are investigating
whether a journal can be trusted. Just take the opposite of their wise
recommendations and make sure to \textbf{not} pick a journal from the list
\href{https://oaspa.org}{Open Access Scholarly Publishers Association} since
these journals have been thoroughly reviewed and can be trusted. For example,
if none of your colleagues know the journal, if the editorial board is made of
unknown or non-existent people \citep{sorokowski:2017} or even dogs in some
cases \citep{kennedy:2017} and no address / email / telephone is provided to
contact the publisher, you may have knocked at the right door. Of course, you
will have to pay exorbitant fees to have your work published, but this is the
price to pay to have an expedited and complaisant review, if any review at
all. Before making you decision where to publish, make sure to check as well
for the retraction fees that may be much higher than the initial publishing
fees in some cases. But who wants to retract anyway?

\section*{Rule 6: Don't give access to your code and data}

%% \begin{wrapfigure}{l}{0.4\textwidth}
%%   \vspace{-6mm}
%%   \begin{center}
%%     \includegraphics[width=0.4\textwidth]{data-share}
%%   \end{center}
%%   \caption{Image by Ainsley Seago.}
%%   \label{fig:datasaurus}
%% \end{wrapfigure}

You managed to have your article published, congratulations! You'll probably
get a lot of buzz and maybe your amazing conclusion will make it to the
newspaper headlines if your research is sexy enough. Because of the buzz,
chances are increased that people will start asking you to access your
material, if the publisher has not already requested it prior to
publication. Of course, that's quite a bad news for you because if your data
has been fabricated or if your statistical analysis is not really academic,
your colleagues might discover your misconduct. Consequently, you definitely
cannot give access to your raw material. For this, you can use the same old
reasons that have prevented data sharing for a very long time
\citep{roche:2014}: ``Oh you know, my data cannot be anonymized'', ``No, you
would not understand the structure of my data'', ``A lot of money has been
invested, I cannot give it for free'', etc. You can also tell these people
asking for your data that they are {\em research parasites} \citep{longo:2016}
even though the techniques seems to have backfired since then
(\url{http://researchparasite.com}).  Technological excuses might also be worth
a try. When \citep{collberg:2014,collberg:2015} tried to access the code of
several publications, they get some really great answers ``Too busy to help'',
``Code will be available soon'', ``Bad backup practice'', etc. Just pick one
you like. Finally, with the increasing pressure by institutions (NIH, NSF,
Europe, etc) to share data, you can pretend (and only pretend) to be a
supporter of Open Data by inserting a simple ``data available upon request'' or
``data is available from my website''. This does not correspond at all to the
FAIR principles (findable, accessible, inter-operable, re-usable,
\cite{wilkinson:2016}), but who cares if you can give the illusion to not be
afraid of sharing your data? But don't get me wrong, you will never answer a
request, and the data will never appear on your website.

\section*{Rule 7: Do not allow for replication outside your lab}

Here comes the tricky part: replications. Can you imagine that some researchers
are willing to replicate your results based on what you've explained in your
article? \citep{nosek:2015} replicated 100 experiments reported in papers
published in three high-ranking psychology journal. And of course, they
concluded that some experiments were not replicable at all. How surprising!
Even worse, there is now scientific journals dedicated to replications
(e.g. \url{http://rescience.github.io}). If people start to replicate your work
and do not get the same results, you have a problem. You might be suspected of
fraud or misconduct and people will really insist on seeing you actual
data. You can try rule 6, but after several failed replication, it will be much
harder to deny access to your data. One easy way to avoid imminent disaster is
to claim beforehand in your article that the experiment/study has been already
replicated in your lab. The subliminal message to the reader is ``Don't bother
to replicate, we've already taken care of that and it works, believe us. Don't
lose your time, you will thank us later''. If this doesn't work, you can also
try a {\em Do Not Replicate order} as proposed by
\href{https://mchankins.wordpress.com/2014/07/12/do-not-replicate/}{Matthew
  Hankins} (who is kind enough as to provide a template). It is not yet widely
used yet, but with the replication crisis we're facing in several scientific
domain, we're confident it will soon become popular.

\section*{Rule 8: Never, ever, retract your results}

If you've made a genuine (and big) mistake in your work, there is no problem in
asking for the retraction of your paper \citep{miller:2006}. This is actually a
behavior that is rewarded according to \citep{lu:2013}, even though in some
cases, it can take several years \citep{trivers:2009}. However, if you've been
frauding, having your paper retracted is like an admission of fraud and you
want to avoid that. Such retraction may start with a simple comment to the
editor saying there may be some problem with you paper. If you're lucky, such
comment, even if sound, will never make it into the journal
\citep{trebino:2009}, else, you'll have to give a convincing answer. Depending
how you addressed the comment, the editor can decide to retract your paper even
without your consent. Before reaching this failure climax, you better have to
start with a simple corrigendum pretexting a bad preparation of the material
for the publication.  Don't hesitate to publish as many corrigendum as
necessary to make critics happy. This can last for several years, the time
needed for people to forget about your fraud or misconduct.


\section*{Rule 9: Don't get caught. Deny if caught.}

If you intend to persist in your rogue scientific career, you have to know that
most likely, you'll get caught. The sooner or the later, but it will happen
eventually. The number of researcher to have never been caught is extremely low
and only the best can hope to be caught only after their death
\citep{degroote:2016,}. But being caught does not mean you career will come to
an end. There is actually a set of simple rules to deny having committed any
scientific misconduct:
\begin{itemize}
  \item If you're first author, explain you were supervised by last author and
    had no choice.
  \item If you're last author, explain you were not aware of the misconduct of
    the first author.
  \item If your name is not first nor last, explain you did not even know your
    name appeared in the publication.
  \item If your name appears because of {\em gift authorship}, just say it.
    It might be considered misconduct, but fraud is much worse.
  \item The intern, that cannot be contacted because he/she left academia is
    responsible for everything
  \item Send threat letter to those who have spotted your misconduct
  \item Sue'em all
\end{itemize}
If you're lucky enough, any of the above will do and things should be back to
normal, especially if you're supported by you employer because you've been
labeled as a rising star.

\section*{Rule 10: Be creative (for once)}

If you look at the yearly list of
\href{http://www.the-scientist.com/?articles.view/articleNo/47813/title/Top-10-Retractions-of-2016/}{top
  ten retractions} compiled by Adam Marcus and Ivan Oransky (co-founders of the
RetractionWatch) since 2012, you'll realize that all the above rules are
already quite well known in the community. If you want to stay off the radar
while committing fraud and misconduct, you'll have to be creative and invent
new rules. 



\section*{Conclusion}

By following the simple rules above, you should get an instantaneous (but
brief) scientific glory, possibly followed by jail time. Nothing comes
for free. Since 2015, scientific misconduct can officially lead you to jail
\citep{grant:2015}. More precisely, a former researcher has been sentenced to
57 months jail and to pay-back 7.2 millions dollars. Science has been and is
still poisoned by fraud and misconduct, but it is now fighting back and the
high-tech war has begun \citep{buranyi:2017}. Today, the risks associated with
fraud and misconduct are really high as well as the chances of being
caught. Consequently, you'd better think twice before committing misconduct or
your name will soon appear in the
\href{https://en.wikipedia.org/wiki/Scientific_misconduct#Notable_individual_cases}{hall of shame}.


% -------------------------------------------------------------- References ---
\renewcommand*{\bibfont}{\small}
\printbibliography[title=References]
\end{document}

